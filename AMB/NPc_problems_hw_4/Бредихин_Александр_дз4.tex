\documentclass[a4paper,12pt]{article} % тип документа

% report, book



%  Русский язык

\usepackage[T2A]{fontenc}			% кодировка
\usepackage[utf8]{inputenc}			% кодировка исходного текста
\usepackage[english,russian]{babel}	% локализация и переносы
\usepackage{graphicx}
\graphicspath{{./}}
\DeclareGraphicsExtensions{.png,.jpg}


% Математика
\usepackage{amsmath,amsfonts,amssymb,amsthm,mathtools} 


\usepackage{wasysym}

%Заговолок
\author{Бредихин Александр}
\title{Домашняя работа №4}



\begin{document} % начало документа
\maketitle
\subsection*{Задача 1}
Определите, являются ли задачи выполнимости и тавтологичности булевой формулы в ДНФ $\mathcal{P}, \mathcal{NP}c$ или $co\mathcal{NP}c$. \smallskip \\

{\bf Выролнимые ДНФ} \\
Пусть $ L $ -- язык выполнимых ДНФ. Покажем, что $ L\in \mathcal{P} $, то есть существует характеристическая функкция $\chi_L(x)$, которая за полиномиальное время проверяет принадлежит слово $ x $ языку или нет.\\
Из дискретного анализа знаем, что ДНФ выполнима, когда выполним хотя бы один из конъюктов. Чтобы конъюкт был выполним нужно, чтобы в нём одновременно не встречалось переменной и её отрицания.\\
Характеристическая функция $\chi_L(x)$ работает следующим образом: циклом проходит по ДНФ, по каждому конъюкту. Если в одном из них нет переменной и её отрицания, то ДНФ выполнима и $\chi_L(x) = 1$. Если после прохождения всей ДНФ функция ничего не вернула, то значит ДНФ не выполнима и $\chi_L(x) = 0$.\\
$\chi_L(x)$ работает за полиномиальное от длины ДНФ время ($\mathcal{O}(n)$). Размер ДНФ, то есть количество конъюктов в нём конечное число, как и переменных в крнъюкте, значит $\chi_L(x)$ -- полиномиальная.\\
Так как $ L\in \mathcal{P} \longrightarrow L \notin \mathcal{NP}c$, $\quad L \notin co\mathcal{NP}c$ (так как $ \mathcal{NP} $ уже нельзя свести к  $ \mathcal{P} $, значит уже не полная)\\
Ответ: $ L\in \mathcal{P}$, $ \quad L \notin \mathcal{NP}c$, $\quad L \notin co\mathcal{NP}c$\\

{\bf Тавтологичные ДНФ} \\
Пусть $ L $ -- язык тавтологичных ДНФ, докажем, что $ L \in co\mathcal{NP}c $, то есть, что язык $ L^* = coL = \{ f: \exists x : f(x) = 0\}$ является $\mathcal{NP}c$.\\
Рассмотрим $ SAT \in \mathcal{NP}c$ -- язык выполнимых КНФ и покажем сводимость $ SAT \leqslant_p L^* $. Рассмотрим такую функцию $ g(x) $: она применяет закон де Моргана к отрицанию КНФ, то есть меняет все конъюкции на дизъюнкции и наоборот. Также меняет переменную на её отрицание и наоборот, отрицание переменной на саму переменную. В итоге получим $ g(x) $ -- ДНФ.\\
\begin{itemize}
\item Если $ x \in SAT $, то по определению существует набор $\exists x: f(x)=1$, тогда $g(x) = 0$. Получается, $ g(x)$ -- ДНФ и нетавтологична (так как есть набор, где она принимает значение 0), значит $g(x) \in L^*$.
\item Если $f \notin SAT$, то $\forall x : f(x) = 0$ (по определению $ SAT $), тогда $\forall x : g(x) = 1$. Получаем, $ g(x) $ -- ДНФ и тавтологична а значит $ g(x) \notin L^* $
\end{itemize} 
Функция $ g(x) $ полиномиальна (аналогично рассужденям для разрешимых ДНФ, проходит по КНФ один раз), поэтому мы доказали сводимость $ SAT \leqslant_p L^* $, следовательно: $ L \in co\mathcal{NP}c $\\
Ответ: $ L \in co\mathcal{NP}c $\\

\subsection*{Задача 2}
{\bf $EXACTLY3SAT \leq_p 3SAT$}
Построим такую функцию сводимости $ f $:
\begin{itemize}
\item если в дизъюнкте 3 литерала или больше, то он не изменяется и остаётся таким же.
\item если в дизъюнкте меньше трёх литералов (то есть дизъюнкты вида $ (x \vee y) $ или $ (x) $), то функция $ f $ заменяет их на дизъюнкты из 4 литералов: $ (x \vee y \vee a \vee b) $, где $ a,b $ - произвольные переменные.
\end{itemize}

Если $ x \in EXACTLY3SAT $, то каждый дизъюнкт $ x $ состоит из трёх литералов и $ f(x)=x $. Для $ x $ есть выполняющий набор, следовательно $f(x) \in 3SAT$ (по определению)\\

Если $ x \notin EXACTLY3SAT $, то возможны случаи: 
\begin{itemize}
\item[1)] все дизъюнкты $ x $ состоят из 3х литералов, но не существует выполняющего набора.
\item[2)] есть дизъюнкт, где больше 3х литералов
\item[3)] есть дизъюнкт, где меньше 3х литералов
\end{itemize}
Для 1го случая $ f(x)=x $ (так как $ f $ не меняет дизъюнкты с 3 литералами), так как для $ x $ нет выполняющего набра, то для $ f(x) $ тоже, значит, $ f(x) \notin 3SAT $.\\
Для 2го случая дизъюнкт, где больше 3х литералов не изменится и останется в $ f(x) $, значит $ f(x) \notin 3SAT$.\\
Для 3го случая дизъюнкты, где меньше 3х литералов перейдут в дизъюнкты с 4 литералами, следовательно, в $ f(x) $ будут дизъюнкты с больше чем тремя литералами, то есть $ f(x) \notin 3SAT $.\\

Функция $ f $ - полнимиальная (работает за 1 проход по входному $ x $), следовательно $EXACTLY3SAT \leq_p 3SAT$.\\
\\
\\
{\bf $3SAT \leq_p EXACTLY3SAT$}\\
Рассмотрим такую функцию $ f = f(x) $:
\begin{itemize}
\item если в дизъюнкте $ x $ 3 литерала, то $ f $ не изменяет его.
\item если в дизъюнкте $ x $ 2 литерала, то есть $(x \vee y)$, то $ f(x) = (x \vee y \vee z) \wedge (x \vee y \vee \neg z) $. Заметим, что первоначальный дизъюнкт равен 1 тогда и только тогда, когда $ f(x) $ равен 1 (от добавленной переменной $ z $ ничего не зваисит).
\item если в дизъюнкте 1 литерал, то есть $(x)$, то $ f(x) = (x \vee y) \wedge (x \vee \neg y) $ а затем применяет пункт 2 к полученным 2м дизъюнктам и получает равносильные дизъюнкты с 3 литералами в каждом.
\end{itemize}
Во всех трёх случаях из принципа работы $ f(x) $ следует, что если $ x\in 3SAT \Leftrightarrow f(x)\in EXACTLY3SAT $. $ f $ -- полиномиальна от длины входного $ x $, значит, $3SAT \leq_p EXACTLY3SAT$.


\subsection*{Задача 3}
Докажите, что задача $VERTEX$-$COVER \in \mathcal{NP}c$\\

Покажем, что задача $VERTEX$-$COVER \in \mathcal{NP}$: в качестве сертификата выберем само вершинное покрытие $ V^* \subseteq V$. Предикат проверяет, что $ |V^*| = k $ а затем для каждого ребра проверяется, что хотя бы одна из его вершин принадлежит $ V^* $ ( это происходит за полиномиальное время, так как количество рёбер - конечное число).\\

Теперь, покажем, что задача $VERTEX$-$COVER$ -- полная, для этого построим сводимость  $CLIQUE \leq_p VERTEX-COVER$.\\
Рассмотрим такую функцию сводимости $ f $: $f((G, k)) = (G^*, k^*)$, где $ G^* $ -- дополнение к графу $ G $ (то есть в нём есть все рёбра, которых нет в $ G $ и наоборот: нет рёбер, которые есть в $ G $). $ k^* = n - k$ ($ n $ -- всего количество вершин в $ G $).\\

Пусть $(G, k) \in CLIQUE$. Пусть $ M $ - множество вершин размером $ k $, образующие клику, $ N $ - все оставшиеся вершины. Тогда у $ f((G, k)) $ у каждого ребра хотя бы одна вершина принадлежит $ N $ (по определению дополнения графа, с учётом того, что в нём есть клика - $ M $), следовательно, вершины $ N $ образуют вершинное покрытие у $ f((G, k)) $ (по определению вершинного покрытия) и его размер $ n-k \longrightarrow f((G, k)) \in VERTEX-COVER$.\\


Пусть $(G, k) \notin CLIQUE$. От противного: пусть в $ f((G, k)) $ есть вершинное покрытие размером $ n-k $. Тогда можно рассмотреть множество вершин, не входящих в него - $ M $, тогда все вершины в $ M $ не связаны друг с другом (иначе вершинное покрытие было бы другое), но тогда в $ (G,k) $ эти вершины образовали бы клику размером $ k $ (по определению дополнению графа). Получили противоречие, следовательно, $f((G, k)) \notin VERTEX-COVER$.

Функция $ f $ -- полиномиальная, так как строит дополнение графа (один раз проходит по матрице смежности). Значит, мы показали сводимость $CLIQUE \leq_p VERTEX-COVER$. Из семинара $CLIQUE \in \mathcal{NP}c \longrightarrow VERTEX$-$COVER \in \mathcal{NP}c$


\subsection*{Задача 4}
Докажите, что задача ПРОТЫКАЮЩЕЕ-МНОЖЕСТВО $\in \mathcal{NP}c$. \\

Пусть $ L $ -- ПРОТЫКАЮЩЕЕ-МНОЖЕСТВО. Покажем, что $ L \in \mathcal{NP}$:\\
Верификатор получает на вход множество $ B $ из $ k $ элементов и проверяет пересечение со всеми множествами $ A_i $, если все пересечения непустые, то возвращается 1, иначе 0. Верефикатор полиномеален, так как множеств $ A_i $ - конечное число и в каждом из них конечное число переменных.\\

Покажем, что $L \in \mathcal{NP}c$, для этого построим сводимость $SAT \leq_p L$ (где $ SAT $ предполагает КНФ).\\
Зададим функцию сводимости $ f = f(x_1, \cdots , x_n) $, где $ x_i $ - переменная SAT, следующим образом: она по SAT строит семейство подмножеств $ A_i $.\\
Сначала строим множества вида $A_i = \{x_i, \neg x_i\}$ для каждой переменной из SAT, затем для каждого дизъюнкта строим $ A_i $ состоящие из всех логических переменных в данном дизъюнкте (то есть если там было отрицание, то ставим отрицание и т.д.)\\

Пусть $ y \in SAT $, следовательно существует её выполняющий набор $ x $. Тогда множество $ B $ которое состоит из таких элементов: если в выполняющем наборе $ x_i = 1 $, то и в $ B $ лежит $ x_i $, если в выполняющем наборе $ x_i = 0 $, то в $ B $ лежит $ \neg x_i $. (то есть в $ B $ ровно $ k $ элементов, количество переменных в SAT).\\
$ B $ является протыкающим множеством для $ A_i $. Докажем это от противного: пусть есть $ A_j $ с которым пустое пересечение. Пусть это множество построено на основе дизъюнкта F. Рассмотрим этот дизъюнкт: если в F содержится $ x_i $, то в $A_j$ содержится $ \neg x_i $ и наоборот (иначе бы было пересечение). Из взятия элементов в $ A_j $ получаем, что все $ x_i $ в дизъюнкте должны равняться 0, следовательно весь дизъюнкт равен 0, тогда получаем противоречие, что $ x $ - выполняющий набор, значит, $ B $ -- протыкающие множество для $ A_i $.\\

Пусть $ y \notin SAT$ (предполагаем, что мы работаем на множестве КНФ, то есть не рассматриваем $ y $, которые не принадлежит КНФ, только те, для которых не находится выполняющего набора), покажем, что для $ A_i $ нет протыкающего множества. От противного, пусть $ B $ -- протыкающее множество для $ A_i $. Понятно, что в $ B $ есть либо $ x_i $ либо $ \neg x_i $ (иначе не было бы пересечений с множествами вида $ A_i = \{x_i, \neg x_i \} $. Для кадого дизъюнкта есть хотя бы одна логическая переменная из $ B $ (иначе не было бы пересечений со множествами 2го вида), но тогда по построению значение это переменной равно 1, следовательно каждый из дизъюнктов равняется 1, получается, что $ y $ -- выполним. Противоречие. Следовательно, для $ A_i $ нет протыкающего набора.\\
Построение $ A_i $ полиномиально, так как в SAT конечное число переменных и дизъюнктов. 

\subsection*{Задача 5}
Покажите, что $VERTEX$-$COVER \leqslant_p SET$-$COVER$. \\

Построим функцию сводимости $ f $ следующим образом:\\
введём обозначения: $ U $ -- множество элементов, а $ S $ это семейство подмножеств $ U $. Пусть $ k $ это такое количество подмножеств из $ S $, таких что их объединение это $ U $.\\
$(G=(V, E), k) \in VERTEX$-$COVER$. Тогда, пусть $U = E$ и функция $ f $ в $ S $ добавляет для всех вершин из $ V $ рёбра такие, что они инцинденты с этими вершинами, то есть 
$S_v = \{e \in E: e$ инцидентно $v\}$ $ \forall v \in V $.\\
Покажем, что $(G=(V, E), k) \in VERTEX$-$COVER \Leftrightarrow f((G, k)) = (U, S, k) \in SET$-$COVER$\\

Пусть $(G=(V, E), k) \in VERTEX$-$COVER$, значит существует $ A $ -- вершинное покрытие графа $ G $ размер которого $ k $, тогда множество $f(G,k) = S_v: v \in A$ образует $ setcover $ для $ U $, так как если мы предположим, что некоторый элемент из $ U \notin S_v$, то в $ A $ не будет вершины, которая бы покрывала это ребро и $ A $, следовательно получили не $ vertexcover $, также размер размер $ S_v $ равен $ k$, так как в $A$ $k$ вершин. Следовательно, $f(G,k) = S_v: v \in A$ -- $ setcover $.\\

В обратную сторону, пусть $(U, S, k) \in SET$-$COVER$, тогда $ A \{v:S_v$ входит в set-cover $U \}$ будет являться vertex-cover размера $k$ для $G$: $f((G, k)) = (U, S, k)$. По построению: все элементы из $ U $ входят в какое-то множество $ S_v \longrightarrow$ все рёбра $ G $ покрыты вершинами из $ A $.\\

Построили $ f $ - полиномиальную, так как количество рёбер и вершин в $ G $ конечно, следовательно $VERTEX$-$COVER \leqslant_p SET$-$COVER$.

\subsection*{Задача 7}
Докажите, что $\Sigma_k \cup \Pi_k \subset \Sigma_{k+1} \cap \Pi_{k+1}$. \\

Для решения задачи нужно показать 4 вложения:
\begin{itemize}
\item[1)] $\Sigma_k \subset \Sigma_{k+1}$
\item[2)] $\Sigma_k \subset \Pi_{k+1}$
\item[3)] $\Pi_k \subset \Pi_{k+1}$
\item[4)] $\Pi_k \subset \Sigma_{k+1}$
\end{itemize}
Из этого и будет значить утверждение задачи.\\
Покажем 1ое и 2ое вложения, 3ие и 4ое делаются аналогично. По определению:
$$
\Sigma_k = x \in A \Leftrightarrow \exists y_{1} \forall y_{2} \exists y_{3} \ldots \forall y_k : V\left(x, y_{1}, y_{2}, \ldots, y_{k}\right)=1
$$
$$
\Sigma_{k+1} = x \in A \Leftrightarrow \exists y_{1} \forall y_{2} \exists y_{3} \ldots \forall y_k \exists y_{k+1} V\left(x, y_{1}, y_{2}, \ldots, y_{k+1}\right)=1
$$
$$
\Pi_{k+1} = x \in A \Leftrightarrow \forall y_{1} \exists y_{2} \exists y_{3} \ldots \exists y_k \forall y_{k+1} : V\left(x, y_{1}, y_{2}, \ldots, y_{k+1}\right)=1
$$
(делаем аналогично контрольной с семинара, только для общего случая с $ k $. Почему в одном случае фиктивная переменная -- последняя, а вдругом первая, обсуждалось на семинаре)\\
Для 1го: $\Sigma_k \subset \Sigma_{k+1}$. 
Пусть $A \in \Sigma_k$, т.е $x \in A \Leftrightarrow \exists y_{1} \forall y_{2} \exists y_{3} \ldots \forall y_k : V\left(x, y_{1}, y_{2}, \ldots, y_{k}\right)=1$. Тогда $x \in A \Leftrightarrow \exists y_{1} \forall y_{2} \exists y_{3} \ldots \forall y_k \exists y_{k+1} : V\left(x, y_{1}, y_{2}, \ldots, y_{k}\right)=1$, где $y_{k+1}$ фиктивная переменная, и предикат $V$ ее не использует. По определению $A \in \Sigma_{k+1}$. Значит $\Sigma_k \subset \Sigma_{k+1}$.\\

Для 2го: $\Sigma_k \subset \Pi_{k+1}$. Пусть $A \in \Sigma_k$, т.е $x \in A \Leftrightarrow \exists y_{1} \forall y_{2} \exists y_{3} \ldots \forall y_k : V\left(x, y_{1}, y_{2}, \ldots, y_{k}\right)=1$. Тогда $x \in A \Leftrightarrow \exists y_{0} \forall y_{1} \exists y_{2} \ldots \forall y_k : V\left(x, y_{1}, y_{2}, \ldots, y_{k}\right)=1$, где $y_0$ фиктивная переменная, и предикат $V$ ее не использует. По определению $A \in \Pi_{k+1}$. Значит $\Sigma_k \subset \Pi_{k+1}$.\\

Аналогично 3) и 4)
\subsection*{Задача 9}
Докажите, что полиномиальная иерархия <<схлопывается>>, если существует $\mathcal{PH}c$ задача. \\
Под схлопыванием имеется в виду $\exists k: \mathcal{PH} = \Sigma_k = \Pi_k$. \\

Пусть язык $A \in PH-\text{полный}$, тогда он лежит и в $ PH $ и поэтому лежит в $ \Sigma_k $ для некоторого $ k $ (по определению $ PH $). Так как $A \in PH-\text{полный}$, то $ \forall B \in PH \longrightarrow B \leq_p A $, значит, $ B $ также лежит в $ \Sigma_k $. Поэтому $ PH = \Sigma_k $ для некоторого $ k $.\\
По определению $ \mathcal{PH} = = \cup \Pi_k = \Sigma_k$, поэтому для любого $ n $ верно $\Pi_{k+n} \subseteq \Sigma_k,$. Из семинара $\Sigma_k \subseteq \Pi_{k+n}$ для любого $ n $. В итоге получаем, что $\Pi_k = \Sigma_k = \mathcal{PH}$ (что и требовалось доказать).
\end{document} % конец документа