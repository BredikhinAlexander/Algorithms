\documentclass[a4paper,12pt]{article} % тип документа

% report, book



%  Русский язык

\usepackage[T2A]{fontenc}			% кодировка
\usepackage[utf8]{inputenc}			% кодировка исходного текста
\usepackage[english,russian]{babel}	% локализация и переносы
\usepackage{graphicx}
\usepackage{tikz}
\graphicspath{{./}}
\DeclareGraphicsExtensions{.png,.jpg}


% Математика
\usepackage{amsmath,amsfonts,amssymb,amsthm,mathtools} 


\usepackage{wasysym}

%Заговолок
\author{Бредихин Александр}
\title{Домашняя работа №9}



\begin{document} % начало документа
\maketitle

\subsection*{Задача 1}
\textit{Задача:} Имеются окрашенные прямоугольные таблички трёх типов: черный квадрат размера $2\times 2$, белый квадрат того же размера и серый прямоугольник $2\times 1$ (последний можно поворачивать на $90^\circ$). Нужно подсчитать число способов $F_n$ замостить полосу размера $2\times n$. Найдите явную аналитическую формулу для $F_n$ и вычислите $F_{30000}$ по модулю $31$. \smallskip \\

Для начала составим рекурентную формулу для нахождения кол-ва замощений полоски: $ F(1) = 1 $ (так как полоску $2\times 1$ можно замостить только серой полоской), $ F(2) = 4 $ (можем поставить 2 квадратика разного цвета или две вертикальные или горизонтальные серые полоски).\\

Заметим, что мы можем ставить одну серую вертикальную полоску и тогда задача сведётся к этой же только размер будет на 1 меньше (то есть $ F(n-1) $), также можем поставить квадратик какого-то цвета или две горизонтальные полоски и тогда задача с ведётся к этой же только размер полоски будет меньше на 2 (не учитываем вариант с двумя вертикальными, так как он получается применением первого случая дважды). Получаем:\\
$$ F(n) = F(n-1) + 3 \cdot F(n - 2) $$
$$ F(1) = 1$$
$$ F(2) = 4$$\\

Нам нужно получить явную аналитическую формулу, для этого воспользуемся методом производящих функций (считаем, что $ F(0) = 1 $ для удобства вывода производящей функции):\\

Пусть $ g(x) = F(1) \cdot x + F(2) \cdot x^2 + \ldots + F(n) \cdot x^n = \sum\limits_{n = 0}^{\infty} F(n)x^n$.\\
Домножим этот ряд на $ x $ и на $ 3x^2 $, получим:\\
$$
xg(x) = \sum\limits_{n = 0}^{\infty} F(n)x^{n+1} = x + F(2)x^2 + \sum\limits_{n = 3}^{\infty} F(n)x^{n+1}
$$
$$
3x^2 g(x) = \sum\limits_{n = 0}^{\infty} 3F(n)x^{n+2} = 3F(1) + \sum\limits_{n = 2}^{\infty} 3F(n)x^{n+2}
$$
Складываем полученные ряды, получаем:\\
$$
g(x) (x+3x^2) = x + \sum\limits_{n = 2}^{\infty} (3F(n-2) + F(n-1))x^{n} = x + g(x) - (F(1) + F(2)x)
$$
После преобразования: 
$$
g(x) = \frac{2x+1}{1-x-3x^2} 
$$
Раскладываем на простые множители полученную функцию:
$$
g(x) = \frac{2x+1}{-3\left(x-\frac{-1+\sqrt{13}}{6}\right)\left(x-\frac{-1-\sqrt{13}}{6}\right)}
$$
Методом неопределённых коэффициентов получаем сумму таких дробей:
$$
g(x) = \frac{1+\frac{2}{\sqrt{13}}}{-3 x+\frac{-1+\sqrt{13}}{2}}+\frac{1-\frac{2}{\sqrt{13}}}{-3 x+\frac{-1-\sqrt{13}}{2}}
$$
Теперь находим коэффициент при $ x^n $ раскладывая слагаемые в ряд используя расширенные биноминальные коэффициенты\\
$$
g(x) = \frac{2+\frac{4}{\sqrt{13}}}{-1+\sqrt{13}\left(1-\frac{6}{-1+\sqrt{13}} x\right)}+\frac{2-\frac{4}{\sqrt{13}}}{-1-\sqrt{13}\left(1-\frac{6}{-1-\sqrt{13}} x\right)}
$$
Получается:
$$
F(n)=\frac{2 \sqrt{13}+4}{-\sqrt{13}+13} \cdot\left(\frac{6}{-1+\sqrt{13}}\right)^{n}+\frac{2 \sqrt{13}-4}{-\sqrt{13}-13} \cdot\left(\frac{6}{-1-\sqrt{13}}\right)^{n}
$$
Полученная формула является явной для $ n $го члена. Преобразуем её: избавимся от иррациональности в знаменателе и приведём подобные слагаемые:\\
$$
F(n)=\frac{(13+\sqrt{13})(2 \sqrt{13}+4)}{169-13}\left(\frac{6+6 \sqrt{13}}{12}\right)^{n}-\frac{(2 \sqrt{13}-4)(13-\sqrt{13})}{156} \left(\frac{-6(\sqrt{13}-1)}{12} \right)^n
$$
$$
F(n)=\frac{30 \sqrt{13}+78}{156}\left(\frac{1+\sqrt{13}}{2}\right)^{n}-\frac{30 \sqrt{13}-78}{156}\left(\frac{1-\sqrt{13}}{2}\right)^{n}
$$
В итоге: 
\begin{equation}
F(n) = \frac{15 \sqrt{13}+39}{78}\left(\frac{1+\sqrt{13}}{2}\right)^{n}-\frac{15 \sqrt{13}-39}{78}\left(\frac{1-\sqrt{13}}{2}\right)^{n}
\end{equation}
Получили похожую задачу семинарской только для другой аналитической формулы. Производим похожие рассуждения:\\

Проверим 13 квадратичный вычет по модулю $p$ или нет (сразу проверим для $ p = 29 $ и $ p = 31 $)\\
для $ p = 31 $ (аналогично примеру из 3ей задачи)
$$
\left(\frac{3}{31}\right)=_{6}\left(\frac{31}{13}\right)=\left(\frac{5}{13}\right)=\left(\frac{13}{5}\right)=\left(\frac{3}{5}\right)=\left(\frac{5}{3}\right)=\left(\frac{2}{3}\right)=-\left(\frac{3}{2}\right) = -\left(\frac{1}{2} \right) = -1
$$
То есть получаем, что по модулю 31 это не квадротичный вычет. Для $ p = 29 $
$$
\left(\frac{13}{29}\right)=\left(\frac{29}{13}\right)=\left(\frac{3}{13}\right)=\left(\frac{13}{3}\right)=\left(\frac{1}{3}\right)=1
$$
Это квадратичный вычет, поэтому можем найти $\sqrt{13}$, подставляем в уравнение и находим $F_n \mod p$ как любую другую формулу используя свойства вычетов и сравнений по модулю (смотреть задачу 2). \\

В этом случае 13 не вычет по модулю 31. Многочлен $x^2 - 13$ неприводим в $\mathbb{Z}_p[x]$, поэтому по нему можно факторизовать, и рассматривать вычеты не просто в виде чисел от $0$ до $p$, а вычеты~---~многочлены степени не более 1 и коэффициентами из $\mathbb{Z}_p$. Такая конструкция называется алгебраическим расширением поля, и также реализована для комплексных чисел: многочлен $i^2+1$ от переменной $i$ неприводим на $\mathbb{R}$, поэтому рассматриваются многочлены степени не более 1, и составляют они привычное $\mathbb{C}$. Операции с многочленами в таком алгебраическом расширении работают так: как только встречаем $x^2$, заменяем на $13$ ($i^2$ на $-1$), и снова остаёмся среди тех же остатков степени не более 1.\\

Получается, заменяем $ \sqrt{13} = x $ находим обратные элементы по модулю 31 к 2 и к 78 (делаем алгоритмом Евклида, как в задаче 3), получаем такую формулу:

$$
F_{n}=(30 x+78)\left(16+16 x\right)^{n}-(30 x-78)\left(16 -16 x\right)^{n}.
$$

Ненулевые многочлены из $\mathbb{Z}_p[x]/(x^2+13)$, которых $31^2-1 = 960$, образуют конечное поле. Это значит, что мультипликативная группа этого поля, то есть поле без нуля (обозначается как $(\mathbb{Z}_p[x]/(x^2+13))^\times$), циклична, то есть существует генератор~---~элемент, который в порождает все остальные своими разными степенями. \\
Всего элементов 960, поэтому, если генератор $g$, $g^{960} = g$, или $g^{960} = 1$. \\
$\forall (ax+b) \ \exists t: g^t = ax+b \implies (ax+b)^{960} = g^{960t} = 1^t = 1$.\\
Ищем для $ k = 30000 $, $k \mod 960 = 240 \mod 960$ \\
Следовательно, $F_k = (30 x+78)\left(16+16 x\right)^{240}-(30 x-78)\left(16 -16 x\right)^{240}$.\\
Считаем $(16\pm 16x)^{240}$ пошаговым возведением в квадрат до 16 степени, получаем $15(\mp x+1)$. Затем возводим этот многочлен в 15ую степень: получили для 2ой и 4ой то есть:\\
$$
19(\pm x+12) \cdot 3(\pm x+20) \cdot 16\left(\mp x-6\right) \cdot 15(\mp x+1) = 9(\pm x -6)
$$
(каждый раз перемножаю многочлены и затем делю с остатком на многочлен $ x^2+13 $ произвожу операции с полученным остатком и так далее).\\


Подставляем полученное выражение в формулу, получаем:
$$
F_k = 9(30 x+78)(x-6)-9(30 x-78)(-x-6)
$$
$$
F_k = \left[ 30 x^{2}-180 x+78 x-468-\left(-30 x^{2}-180 x+78 x+468\right)\right]
$$
$$
F_k = 9\left[60 x^{2}-936\right] = - 9 \cdot 156 \mod 31 = 22
$$
Ответ: 22 (когда пересчитывал ещё раз, получилось 1)


\subsection*{Задача 2}
\textit{Задача:} Решите предыдущую задачу по модулю $29$. \smallskip \\

Мы выяснили, что 13 является квадратичным вычетом по модулю 29. Найдём $ \sqrt{13}$ решив уравнение: $ x^2 = 13 \mod 29 $ решаем аналогично 3ей задаче, получаем $ x= 10, 19 $. Подставляем эти значения в уравнение (1), получаем:\\
$$
\frac{15 \cdot 19+39}{78}(10)^{30000}-\frac{15 \cdot 19-39}{78}(9)^{30000} = 10^{30000} - 9^{30000} = 25 \mod 29
$$
(остаток ищем пользуясь малой теоремой Ферма)\\

Ответ: 25\\
P.S. подставил значение $ x = 10 $ и почему-то получил другой ответ (28), перепроверил, вроде бы нигде не ошибаюсь, но должен же быть один остаток так как $F_{30000}$ - конкретное число, у которого может быть 1 остаток при делении на 29. В этом не смог до конца разобраться. 


\subsection*{Задача 3}
\textit{Задача:}\\

а) Делится ли $4^{1356}-9^{4824}$ на $35$? Делится ли $5^{30000} - 6^{123456}$ на $31$?\\

\textit{Решение:} заметим, что $ 35 = 5 \cdot 7 $, а $ 4^{1356}-9^{4824} = 2^{2712} - 3^{9648} $. Применяем малую теорему Ферма (так как 2, 3 - простые числа) для 5 и для 7:\\
$$
2^4 = 1 (\mbox{mod } 5) \quad \text{так как } 4|2712 \rightarrow 2^{2712} = 1 (\mbox{mod } 5)
$$
$$
3^4 = 1 (\mbox{mod } 5) \quad \text{так как } 4|9648 \rightarrow 3^{9648} = 1 (\mbox{mod } 5)
$$
Получили равные остатки, при вычитании получим 0, то есть наша разность делится на 5: $ 4^{1356}-9^{4824} = 0 (\mbox{mod } 5) $, покажем, что она будет делиться и на 5 аналогично:
$$
2^6 = 1 (\mbox{mod } 7) \rightarrow 2^{2712} = 1 (\mbox{mod } 7)
$$
$$
3^6 = 1 (\mbox{mod } 7) \rightarrow 3^{9648} = 1 (\mbox{mod } 7)
$$
Получили, что $ 4^{1356}-9^{4824} = 0 (\mbox{mod } 7) $. То есть делится и на 5 и на 7, следовательно, делится на 35.\\

Ответ: делится\\

Рассмотрим: $5^{30000} - 6^{123456}$ на $31$. Тут $31$ - простое число, поэтому можно снова применить малую теорему Ферма:\\
$$
5^{30} = 1 (\mbox{mod } 31) \rightarrow 5^{30000} = 1 (\mbox{mod } 31)
$$
$$
6^{30} = 1 (\mbox{mod } 31) \rightarrow 6^{123450} = 1 (\mbox{mod } 31)
$$
Но у нас степень 6ки -- $ 123456 $: поэтому нам нужно остаток от деления: $ 6^{123456} (\mbox{mod } 31) = 6^6 (\mbox{mod } 31) = 1 $.
Получили равные остатки, при вычитании получится 0, следовательно, эта разность тоже делится.\\

Ответ: делится \\

б) Найдите обратные $20 \ (\mbox{mod } 79)$, $3 \ (\mbox{mod } 62)$.\\

\textit{Решение:} с помощью алгоритма Евклида решаем уравнения:\\
$$
20a +79b = 1
$$
\begin{center}
\begin{tabular}{|c|c|c|c|}
\hline 
79 & 20 & 3 & 19 \\ 
\hline 
20 & 19 & 1 & 1 \\ 
\hline 
19 & 1 & 19 & 0 \\ 
\hline 
\end{tabular} \\
\end{center}

$$
1 = 20 -19 = 20 - (79 - 3 \cdot 20) = 4 \cdot 20 - 79
$$

Получается, что $ a = 4 $, $ b = 1 $. Сделовательно:\\

Ответ: $ 4 = 20^{-1} (\mbox{mod } 79) $\\

Найдём:  $ 3^{-1} (\mbox{mod } 79) = \quad ? $\\

$$ 3a + 62b = 1 $$
\begin{center}
\begin{tabular}{|c|c|c|c|}
\hline 
62 & 3 & 20 & 2 \\ 
\hline 
3 & 2 & 1 & 1 \\ 
\hline 
2 & 1 & 2 & 0 \\ 
\hline 
\end{tabular} \\
\end{center}
$$
1 = 3 - 2 = 3 - (62 - 3 \cdot 20) = 21 \cdot 3 - 62
$$
Получается, что $ a = 21 $\\

Ответ: $ 21 = 3^{-1} (\mbox{mod } 79) $\\


в) Найдите все решения уравнения $35x = 10 \ (\mbox{mod } 50)$.\\

\textit{Решение:} для этого решим с помощью алгоритма Евклида такое диофантово уравнение:\\
$$
35x + 50y =10
$$
разделим на НОД коэффицинтов в левой части, получим:
$$
7x + 10y = 2
$$
По алгоритму Евклида:
\begin{tabular}{|c|c|c|c|}
\hline 
10 & 7 & 1 & 3 \\ 
\hline 
7 & 3 & 2 & 1 \\ 
\hline 
3 & 1 & 3 & 0 \\ 
\hline 
\end{tabular} \\
Получается:
$$
1 = 7 - (2 \cdot 3) = 7 - 2 \left(10 - 7 \right) = 3 \cdot 7- 2 \cdot 10
$$
$$
2 = 6 \cdot 7 - 4 \cdot 10
$$
Частное решение: $ x_0 = 6 $, тогда из курса алгоритмов общее решение записывается как:
$ x = 6 + 10k$, где $ k $ - целое.\\

Ответ: $ x = 6 + 10k$\\

г) Имеет ли решение сравнение $x^2 = 1597 \mod 2011$\\

\textit{Решение:}  для этого надо найти $ \left(\frac{1597}{2011}\right) $

$$
\left(\frac{1597}{2011}\right)=_6\left(\frac{2011}{1597}\right)=_1\left(\frac{414}{1597}\right)=_5\left(\frac{2}{1597}\right)\left(\frac{9}{1597}\right)\left(\frac{23}{1597}\right)=_3 
$$

$$
=_3-\left(\frac{3}{1597}\right)\left(\frac{3}{1597}\right)\left(\frac{23}{1597}\right)=_6-\left(\frac{1597}{3}\right)\left(\frac{1597}{3}\right)\left(\frac{1597}{23}\right)=_1 
$$
$$
=_{1}-\left(\frac{1}{3}\right)\left(\frac{1}{3}\right)\left(\frac{10}{23}\right)=_2-\left(\frac{10}{23}\right)=_5-\left(\frac{2}{23}\right)\left(\frac{5}{23}\right)=_3 \\
$$
$$
=_3-\left(\frac{5}{23}\right)=_6-\left(\frac{23}{5}\right)=_1-\left(\frac{3}{5}\right)=_6-\left(\frac{5}{3}\right)=_1-\left(\frac{2}{3}\right)=_3 1
$$

Получили 1, то есть 1597 квадратичный вычет, поэтому решения для этого уравнения есть.\\
Ответ: имеет \\

д) Найдите наименьшее натуральное число, имеющее остатки $2$, $3$, $1$ от деления на $5$, $13$ и $7$ соответственно. \smallskip \\

\textit{Решение:} составим систему сравнений и будем решать её с помощью КТО:\\
$$
\left\{\begin{array}{ll}
x=2 & \bmod 5 \\
x=3 & \bmod 13 \\
x=1 & \bmod 7
\end{array}\right.
$$
$ M = 5 \cdot 13 \cdot 7 $, тогда согласно КТО:\\
$$
\begin{aligned}
&x=2 \cdot \frac{455}{5}\left(\left(\frac{455}{5}\right)^{-1} \bmod 5\right)+3 \cdot \frac{455}{13}\left(\left(\frac{455}{13}\right)^{-1} \bmod 13\right)\\
&+1 \cdot \frac{455}{7}\left(\left(\frac{455}{7}\right)^{-1} \bmod 7\right)=
=182 (91^{-1} \bmod 5)+105\left(35^{-1} \bmod 13\right)+ \\
&+65\left(65^{-1} \bmod 7\right)=182+105 \cdot 3+65 \cdot 4 = 757 \bmod 455 = 302
\end{aligned}
$$
Обратные элементы нахожу с помощью алгоритма Евклида аналогично пункту б. Конечный результат беру по модулю 455 так как в задании требуется наименьшее число.\\

Ответ: 302



\subsection*{Задача 4}
\textit{Задача:} Найти все генераторы для $(\mathbb{Z}/19\mathbb{Z})^\times$.\\

Из семинара: $ x $ является генератором, если: $x^{p-1} = 1 \mod p$, $x^{\frac{p-1}{p_i}} \neq 1 \mod p$. В нашем случае получаем следующую систему: 
$$
\left\{\begin{array}{l}
x^{18} = 1 \mod 19 \\
x^{3} \neq 1 \mod 19 \\
x^{6} \neq 1 \mod 19
\end{array}\right.
$$
С помощью неё проверяем для каждого числа от 2 до 19 эти условия и определяем генераторы группы. Например, для 10: $ 10^18 = 1 \mod 19 \quad 10^9 = 18 \neq 1 \quad 10^6 = 11 \neq 1$, следовательно, 10 это генератор. Аналогичным образом проверяем и получаем\\

Ответ: 2, 3, 10, 13, 14, 15



\subsection*{Задача 5}
\textit{Задача:} Предложите полиномиальный алгоритм нахождения количества натуральных решений диофантова уравнения $ax+by = c$. \smallskip \\

Рассмотрим самый общий случай, когда не один из коэффициентов не равен 0 и НОД$(a,b) | c$. Тогда диофантово уравнение в целых числах имеет бесконечно много решений и они записываются с использованием алгоритма Евклида, как:\\
$x = x_0 + b'k$\\
$y = y_0 - a'k$\\
Где $ k $ -- целое число, а $ a', b' $ -- коэффициенты после деления на НОД$(a,b)$.\\
Из курса алгоритмов знаем, что алгоритм Евклида работает за полиномиальное время от длины входа (то есть от длины  битовой записи числа). Понятно, что количество решений в натуральных числах конечно, так как из формулы решения видно при увелечении $ x $, значение $ y $ уменьшается и когда-то станет отрицательным также работает наоборот для $ x $.\\ 

Чтобы определить количество натуральных решений сначала найдём наименьший $ x > 0 $ (то есть найдём те $ k $ для которых значение $ x $ натурально). Это мы сможем сделать за $ O(1) $, так как мы знаем смещение и частное решение.\\

Сделаем аналогично для $ y $ и найдём тот промежуток для $ k $ где решения $ y $ принимают натуральные значения (также делаем за $ O(1) $). Пересекаем полученные промежутки и получаем отрезок и количество целых чисел в нём и есть количество натуральных решений по построению (и $ x $ и $ y $ будут натуральными) (пересечение работает за полиномиальное время).\\ Получили алгоритм, находящий количество натуральных решений за полиномиальное время. 




\subsection*{Задача 6}
\textit{Задача:} Пусть язык $L\in\mathcal{NP}$. Покажите, что он полиномиально сводится (по Карпу) к языку $STOP$ описаний пар $(M, \omega)$ машин Тьюринга и входов таких, что $M$ останавливается на входе $\omega$. \\

\textit{Решение 1:} построим функцию полиномиальной сводимости $ f $ следующим образом: так как $L\in\mathcal{NP}$ то она решается с помощью сертефиката с предикатом, поэтому функция $ f(w) = (M,w) $ это МТ, которая подставляется все сертефикаты в предикат и проверяет, чему он будет равен. Если он будет равен 1, то она возвращает пару из МТ и слова, на котором она останавливается (<<создаёт>> эту МТ сама). Если при подставлении сертефикатов предикат будет равен 0, то возвращает пару  из МТ и слова, на котором она \textbf{НЕ} останавливается. Получаем:\\
$$
w \in L \rightarrow f(w) = (M,w) - \text{останавливается} \rightarrow f(w) \in STOP
$$
$$
w \notin L \rightarrow f(w) = (M,w) - \text{не останавливается} \rightarrow f(w) \notin STOP
$$
Получили определение полиномиальной сводимости: функция работает за полином, так как размер сертефиката и вычисление предиката полиномиальны.\\

\textit{Решение 2:} знаем, что любая $L\in\mathcal{NP}$ полиномиально сводится к задаче $ 3SAT $. Возбемём МТ, которая будет перебирать все возможные значения набора в $ 3SAT $, если она находит выполняющий, то останавливается, если нет, то работает бесконечно долго, то есть эта МТ остановится тогда и только тогда, когда будет найден выполняющий набор.\\

Так как размер $ 3SAT $, конечный то мы можем свести эту задачу к $ STOP $ за полиномиальное время (так как сведение к $ 3SAT $ происходит за полиномиальное время, и передача на вход МТ, так как размер формулы конечен тоже)\\


\subsection*{Задача 7}
\textit{Задача:} Постройте NP-сертификат простоты числа $p = 3911$, $g = 13$. Известными простыми считаются только числа $2$, $3$, $5$.\\

В конце семинара был показан алгоритм построения сертефиката для проверки простоты числа: он состоит из генератора циклической группы для простого числа, разложение $p-1$ на простые множители и рекурсивно сертификаты для множителей $p-1$. Построим его для нашего простого числа: генератор для каждой группы будем находить из условий $x^{p-1} = 1 \mod p$, $x^{\frac{p-1}{p_i}} \neq 1 \mod p$. Поэтому:\\
\begin{itemize}
\item $ p = 3910, \quad g = 13, \quad p - 1 = 3910 = 2 \cdot5 \cdot 17 \cdot 23  $. 2 и 5 известно, что простые, поэтому нужно делать аналогичные сертефикаты для 17 и 23
\item $ p = 17, \quad p - 1 = 16 = 2^4  $\\
$ g = 3 $, тогда $ 3^{\frac{17-1}{2}} = 2^8 \neq 1 \mod 17, \quad 3^{16} = 1 \mod 17 $ следовательно, $ g = 3 $ -- генератор.
\item $ p =  23, \quad p-1 = 2 \cdot 11$. Аналогично находим, что генератор $ g = 5 $. Появилось 11, для него проделываем аналогичные действия.
\item $ p = 11, \quad p-1 = 2 \cdot 5 $. Генератор: $ g = 2 $.
\end{itemize}
В итоге получаем такой сертефикат (значения идут в порядке описаных выше рассуждений):\\

Ответ: 13, 2, 5, 17, 23, 3, 2, 5, 2, 11, 2, 2, 5

\end{document}





\end{document} % конец документа