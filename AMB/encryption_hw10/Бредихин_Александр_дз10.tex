\documentclass[a4paper,12pt]{article} % тип документа

% report, book



%  Русский язык

\usepackage[T2A]{fontenc}			% кодировка
\usepackage[utf8]{inputenc}			% кодировка исходного текста
\usepackage[english,russian]{babel}	% локализация и переносы
\usepackage{graphicx}
\usepackage{tikz}
\graphicspath{{./}}
\DeclareGraphicsExtensions{.png,.jpg}


% Математика
\usepackage{amsmath,amsfonts,amssymb,amsthm,mathtools} 


\usepackage{wasysym}

%Заговолок
\author{Бредихин Александр}
\title{Домашняя работа №10}



\begin{document} % начало документа
\maketitle

\subsection*{Задача 1}
\textit{Задача:} В протоколе $RSA$ выбраны $p = 17$, $q = 23$, $N=391$, $e=3$. Выберите ключ $d$ и зашифруйте сообщение $41$. Затем расшифруйте полученное сообщение и убедитесь, что получится исходное $41$. \smallskip \\

Передаём сообщение $ m = 41 $ (не надо разбивать на блоки). Найдём закрытый ключ: по алгоритму $RSA$: $ d = e^{-1} \mod (p-1)(q-1) $. Находим обратный элемент с помощью алгоритма Евклида, как в предыдущем семинаре: $ d = 3^{-1} \mod 352 = 235 $.\\

Отправка сообщения: $y = m^e \mod N = 41^3 \mod 391 = 105$ (считаем с помощью КТО аналогично прошлому семинару задаче 3).\\

К получателю приходит зашифрованное сообщение $ y $. Он знает закрытый ключ $ d $. Для расшифровки делает следующее: $ y^d \mod N = 105^{235} \mod 391 = 41 $. Получили то сообщение, которое было зашифровано, что и требовалось показать.

\subsection*{Задача 2}
\textit{Задача:} Пусть в протоколе $RSA$ открытый ключ $(N, e)$, $e=3$. Покажите, что если злоумышленник узнаёт закрытый ключ $d$, то он может легко найти разложение $N$ на множители. \smallskip \\

По определению: $ d = e^{-1} \mod (p-1)(q-1) $, следовательно, $ de = 1 \mod (p-1)(q-1) $. Решаем диофантово уравнение $ de + 1 \cdot (p-1)(q-1) = 1 $ относительно $ e $ и $ (p-1)(q-1) $. Находим $(p-1)(q-1)$ (за полиномиальное время). Делаем следующее:\\
$$
de = 1 \mod (p-1)(q-1)
$$
$$
de + (p-1)(q-1) = 1 \mod (p-1)(q-1) 
$$
$$
de + N - p - q = 0 \mod (p-1)(q-1)
$$
Так как мы знаем $ (p-1)(q-1) $, $ d$, $ e $, то из этого сравнения мы сможем найти $ p+q = E$. Также мы знаем, что $ pq = N $ (значение открытого ключа мы тоже знаем). Получается $ p,q $ - корни квадратного уравнения с коэффициентам $ a = 1 $, $ b = -E $, $ c = N $. Его решение мы находим за полиномиальное время (решение через дискриминант). Поэтому за полиномиальное время, мы можем получить разложение $ N $ на множители.

\subsection*{Задача 3}
\textit{Задача:} Докажите, что в шифре Шамира в итоге у $B$ в действительности оказывается то сообщение, которое $A$ планировал передать.\smallskip \\

Поэтапно пройдём по действиям алгоритма шифрования и убедимся в этом (использую обозначения, как в семинаре):
\begin{itemize}
\item[1) ] A $ \rightarrow $ Б: $ M^{c_A} \mod p$\\
\item[2) ] Б $ \rightarrow $ А: $ \left( M^{c_A} \right)^{c_B} \mod p$.\\
Для следующего пункта учитываем, что $ c_Ad_A = 1 \mod p-1 \Rightarrow c_Ad_A = 1 + n(p - 1) $, где $ n $- натуральное число.\\
\item[3) ] A $ \rightarrow $ Б: $ \left( M^{c_A c_B} \right)^{d_A} = M^{c_B (1 + n(p - 1))} = M^{c_B} \cdot M^{n(p - 1))c_B} \Rightarrow \text{по Малой теореме Ферма} \\ M^{p-1} = 1 \mod p \Rightarrow \left( M^{c_A c_B} \right)^{d_A} = M^{c_B} \left( M^{nc_B} \right)^{p-1} \mod p = M^{c_B} \mod p$

\item[4) ] Б: $ \left( M^{c_B} \right)^{d_B} \mod p $. Так как по Малой теореме Ферма $M^{p-1} = 1 \mod p$, то показатель можно брать по модулю $ p-1 $, учитывая, что $ c_B d_B = 1 \mod p-1 $ (по алгоритму шифрования), получаем:\\
$ \left( M^{c_B} \right)^{d_B} \mod p = M \mod p$. Получили то сообщение, которое и было зашифровано, что и требовалось показать.

\end{itemize}


\subsection*{Задача 4}
\textit{Задача:} Докажите, что в шифре Эль-Гамаля в итоге у $B$ в действительности оказывается то сообщение, которое $A$ планировал передать. \smallskip \\


Используем все те же обозначения, что и в семинаре.\\
Из семинара получаем $ m' = ed^{p-1-c_B}_A $. Покажем, что полученное сообщение совпадает с отправленным.\\
Так как по малой теореме Ферма $ d_A^{p-1} = 1 \mod p $  и из алгоритма шифрования $ e = md_B^{c_A} $, получаем, что $ m' = m d_B^{c_A} \left( d_A^{c_B} \right) ^{-1} \mod p$.\\

Из алгоритма шифрования следует:\\
$ \left( d_A^{c_B} \right) ^{-1} \mod p = \left( g^{c_Ac_B} \right) ^{-1} \mod p$\\
$ d_B^{c_A} \mod p = g^{c_B c_A} \mod p $\\

В итоге получаем: $$ m' = m d_B^{c_A} \left( d_A^{c_B} \right)^{-1}\mod p = m g^{c_B c_A} \left( g^{c_Ac_B} \right) ^{-1} \mod p = m$$
Получили зашифрованное сообщение, что и требовалось показать.



\subsection*{Задача 5}
\textit{Задача:} Докажите, что в алгоритме шифрования Рабина $B$ в итоге сможет найти исходное передаваемое сообщение среди $(\pm apm_q \pm bqm_p)$. \smallskip \\

Используем все обозначения, как в семинаре.\\
Применяем КТО к зашифрованному сообщению: $ M = pq $, $ m_1 = p $, $ m_2 = q $.
Обозначим, что $ m^2 = x_1 \mod p$, $ m^2 = x_2 \mod q $, тогда\\
$$
y = m^2 = x_1 q \left( q^{-1} \mod p \right) + x_2 p \left(p^{-1} \mod q \right) \mod pq
$$

Заметим, что обратные элементы к $ q $ по модулю $ p $ и к $ p $ по модулю $ q $, находятся из решения диофантового уравнения: $ ap + bq = 1$. В итоге получаем такое сравнение:
$$
y = m^2 = x_1 q b + x_2 p a \mod pq
$$

Для нахождения ответа нам нужно извлечь корень из полученного выражения. Заметим, что число $ pq $ -- число Блюма (по определению) для него $ x = y ^{\frac{p+1}{4}} \mod p,q $ - корень из $ k $ по модулю $ p,q $ (так как $ y^{\frac{p+1}{2}} = 1 \mod p,q $). Также (аналогично) $ -y ^{\frac{p+1}{4}} \mod p,q  $ -- тоже корень. Поэтому мы получаем 4 разных варианта $ \pm a p y^{\frac{q+1}{4}} \pm b q y^{\frac{p+1}{4}} \mod pq$. По принципу кодирования, один из этих решений и будет являться первоначальным сообщением.



\subsection*{Задача 6}
\textit{Задача:} Докажите формулу обращения: $(M_n(\omega))^{-1} = \dfrac{1}{n}M_n(\omega^{-1})$. Вычислите также матрицу $(M_n(\omega))^4$. \smallskip \\

Для доказательства утверждения и для нахождения $(M_n(\omega))^4$ докажем такую лемму <<о суммировании>>:\\

Формулировка: для любого целого $ n \geqslant 1 $ и ненулевого $ k $ не кратного $ n $ выполнено:
$$\sum_{j=0}^{n-1}\left(\omega_{n}^{k}\right)^{j}=0$$

Доказательство: используя формулу геометрической прогрессии, получаем: 
$$\sum_{j=0}^{n-1}\left(\omega_{n}^{k}\right)^{j}=\frac{\left(\omega_{n}^{k}\right)^{n}-1}{\omega_{n}^{k}-1}=\frac{\left(\omega_{n}^{n}\right)^{k}-1}{\omega_{n}^{k}-1}=\frac{(1)^{k}-1}{\omega_{n}^{k}-1}=0$$
Так как $ k $ не делит $ n $, то знаменатель в ходе доказательства не равен 0 ($ \omega_{n}^{k} = 1 $ тогда и только тогда, когда $ k $ делится на $ n $).\\

Для доказательства, что $(M_n(\omega))^{-1} = \dfrac{1}{n}M_n(\omega^{-1})$, покажем, что $ M_n^{-1} \cdot M_n = E $ (единичной матрице). Рассмотрим $ (j, j') $ элемент матрицы $ M_n^{-1} \cdot M_n$

$$\left[M_{n}^{-1} M_{n}\right]_{j j \prime}=\sum_{k=0}^{n-1}\left(\omega_{n}^{-k j} / n\right)\left(\omega_{n}^{k j \prime}\right)=\sum_{k=0}^{n-1} \omega_{n}^{k\left(j^{\prime}-j\right)} / n$$

Согласно доказанной ранее лемме, полученная сумма равна 1, если $ j' = j $ и 0 иначе. Так как $-(n-1) \leq j^{\prime}-j \leq n-1, j^{\prime}-j$, то условие леммы выполняется и $ k $ не делит $ n $. Получили единичную матрицу, следовательно, $(M_n(\omega))^{-1} = \dfrac{1}{n}M_n(\omega^{-1})$\\

Найдём $(M_n(\omega))^4$.\\

Возьмём $ i $ую строчку (индексируем с нуля для удобства). $ i $ая строчка состоит из таких элементов:\\ $ 1 \quad w^i \quad w^{2i} \ldots \quad w^{i(n-1)}$\\
Аналогично рассматриваем $ j $ столбец:\\
$ 1 \quad w^j \quad w^{2j} \ldots \quad w^{j(n-1)}$\\ 

При возведении матрицы в квадрат: при перемножении $ i $ой строчки и $ j $го столбца получаем элемент:\\
$ 1 + w^{i+j} + w^{2(i+j)} + \ldots + \quad w^{(i+j)(n-1)}$\\
По лемме о суммировании он равен 0 если $ i + j \neq n+1 $, если $ i + j = n+1 $, то получаем сумму из $ n $ единиц. Получается, что при возведении такой матрицы в квадрат получим такую матрицу:
\begin{equation}
\begin{pmatrix}
n & 0 & 0 & \quad & 0\\
0 & 0 & 0 & \quad & n\\
\quad \\
0 & 0 & n & \quad & 0\\
0 & n & 0 & \quad & 0
\end{pmatrix}
\end{equation}
При перемножении двух таких матриц получаем диагональную с $ n^2 $ в диагонале (перемножаем по определению замечаем закономерность).\\

Ответ: диагональная с $ n^2 $ в диагонале.










\end{document} % конец документа