\documentclass[a4paper,12pt]{article} % тип документа

% report, book



%  Русский язык

\usepackage[T2A]{fontenc}			% кодировка
\usepackage[utf8]{inputenc}			% кодировка исходного текста
\usepackage[english,russian]{babel}	% локализация и переносы
\usepackage{graphicx}
\graphicspath{{./}}
\DeclareGraphicsExtensions{.png,.jpg}


% Математика
\usepackage{amsmath,amsfonts,amssymb,amsthm,mathtools} 


\usepackage{wasysym}

%Заговолок
\author{Бредихин Александр}
\title{Домашняя работа №3}



\begin{document} % начало документа

\maketitle

\subsection*{Задача 1}
{\bf a)} Решить уравнения в целых числах, используя расширенный алгоритм Евклида: $ 238x +385y = 133 $ (общий вид уравнения: $ a \cdot x + b \cdot y = c $) \\

Найдём $ NOD(a,b) $ с помощью обычного алгоритма Евклида и проверим, делится ли на него коэффициент $c = 133 $ или нет (если не делится, то уравнение не имеет решение)\\

\begin{tabular}{|c|c|c|c|}
\hline 
a & b & a//b & a$\%$b \\ 
\hline 
385 & 238 & 1 & 147 \\ 
\hline 
238 & 147 & 1 & 91 \\ 
\hline 
147 & 91 & 1 & 56 \\ 
\hline 
91 & 56 & 1 & 35 \\ 
\hline 
56 & 35 & 1 & 21 \\ 
\hline 
35 & 21 & 1 & 14 \\ 
\hline 
21 & 14 & 1 & 7 \\ 
\hline 
14 & 7 & 2 & 0 \\ 
\hline 
\end{tabular} \\

Получается, $ NOD(a,b) = 7 $. Коэффициент $ c $ делится на $ 7 $ $ \longrightarrow $ Разделим все коэффициенты уравнения на $ NOD(a,b) = 7 $, получится уравнение: 
$ 34x + 55y = 19 $. Применяем к полученному уравнению расширенный алгоритм Евклида:\\

\begin{tabular}{|c|c|c|}
\hline 
$x$ & $y$ & $34x + 55y$ \\ 
\hline 
0 & 1 & 55 \\ 
\hline 
1 & 0 & 34 \\ 
\hline 
-1 & 1 & 21 \\ 
\hline 
2 & -1 & 13 \\ 
\hline 
-3 & 2 & 8 \\ 
\hline 
5 & -3 & 5 \\ 
\hline 
-8 & 5 & 3 \\ 
\hline 
13 & -8 & 2 \\ 
\hline 
-21 & 13 & 1 \\ 
\hline 
\end{tabular} \\

Отсюда получаем: $ x^* = -21, y^* = 13 $, тогда частное решение будет иметь вид:\\
$ x_0 = x^* \cdot \frac{c}{NOD(a,b)} = -21 \cdot 19 = -399 $\\
$ y_0 = y^* \cdot \frac{c}{NOD(a,b)} = 13 \cdot 19 = 247 $ \\
Откуда получаем общее решение уравнения: \\
Ответ: $ x=-399+55k $, $ y=247-34k $, где $ k\in \mathbb{Z} $\\

{\bf a)} Уравнение $ 143x + 121y = 52 $. Найдём $ NOD(a,b) =  NOD(143,121) $ с помощью алгоритма Евклида:\\

\begin{tabular}{|c|c|c|c|}
\hline 
$a$ & $b$ & $a//b$ & $a\%b$ \\ 
\hline 
143 & 121 & 1 & 22 \\ 
\hline 
121 & 22 & 5 & 11 \\ 
\hline 
22 & 11 & 2 & 0 \\ 
\hline 
11 & 0 &  &  \\ 
\hline 
\end{tabular} \\

Значит $ NOD(a,b)=11$. Коэффициент $ c = 52 $ не делится на 11, следовательно, уравнение не имеет решения.\\
Ответ: нет решения.

\subsection*{Задача 2}
Вычислите $ 7^{13} \mod 167$, используя алгоритм быстрого возведения в степень.\\

Применяем алгоритм: $ 13_{10} = 1101_{2} $ $ \longrightarrow $\\
$ 7^{13_{10}}=7^{1101_{2}}=7\left(7^{110_{2}}\right)^{2}=7\left(\left(7^{11_{2}}\right)^{2}\right)^{2}= 
7\left(7 \cdot 7^{2}\right)^{4}=7(7 \cdot 49)^{4}=[\bmod 167]= 7 \cdot(81)^{2}=[\bmod 167]=7 \cdot 48=[\bmod 167] = 2 $ \\
Ответ: 2\\


\subsection*{Задача 3}
Докажем корректность даного рекурсивного алгоритма по индукции по $ x $ (то есть на каждом шаге рекурсии возвращаемая пара $ (q, r) $ -- верная):\\
База индукции: $ x = 0 \longrightarrow (q,r) = (0,0)$ -- верно.\\
Предположение индукции: пусть для пары $ \left(\left[\frac{x}{2}\right], y\right) $ -- алгоритм вернул верную пару $ (q,r) $\\
Шаг индукции: покажем, что для пары $ (x,y) $ получается верный ответ.  \\
Из П.И. $ \left[\frac{x}{2}\right] = q \cdot y + r $ и $ r < y $. Домножим это равенство на 2, получается:
\begin{equation}
\label{aa}
2 \cdot \left[\frac{x}{2}\right] = 2qy + 2r 
\end{equation} 

\begin{itemize}
\item[1] Если $ x $ -- чётное. Тогда уравнение \eqref{aa} принимает вид $ x = 2qy + 2r $. Пусть $ q^* = 2q$  $r^* = 2r $, так как по П.И. $ r < y \longrightarrow 2r $ не может превышать $ y $ больше, чем в 2 раза (следовательно, можно не брать остаток по модулю, а просто вычитать). Если $ r^* \geq y $, тогда нужно проделать следующие операции: $ r^* = r^* - y ; q^* = q^* + 1 $. Теперь $ r^* < y $, значит, пара $ (q^*, r^*) $ является ответом (что и делает наш алгоритм) -- верно.
\item[2] Если $ x $ -- нечётное. Тогда $ 2 \cdot \left[\frac{x}{2}\right] = x - 1 $, следовательно уравнение \eqref{aa} принимает вид $ x = 2qy + 2r + 1 $. Если $ 2r + 1 \geq y $, то получаем (аналогичные рассуждения предыдущему пункту):
$$
x = \left(2q + 1 \right)y + \left( \left(2r + 1 \right) - q \right)
$$
И пара $(\left(2q + 1 \right), \left( \left(2r + 1 \right) - q \right) )  $ - является ответом. Верно.
\end{itemize}
Алгоритм выводит верные пары, следовательно, корректность доказана по индукции.\\

Сложность алгоритма:\\
Числа записаны побитово ($ n $ битов каждое число, в худшем случае), следовательно при вызове $\left[\frac{x}{2}\right]$ у числа убирается 1 бит. Составим рекуренту этого алгоритма:
$$
T(n) = T(n - 1) + n 
$$
На каждом шаге $ n $ операций (побитовое сложение чисел длинной  $ n $ бит). Решая рекуренту аналогично задаче 4, получим $\mathcal{O}(n^2)$\\
Ответ: $\mathcal{O}(n^2)$





\subsection*{Задача 4}
{\bf 1)} $T_{1}(1)=T_{1}(2)=T_{1}(3)$\\
Пошагово раскрываем рекуренту и пользуемся формулой арифметической прогрессии:\\
$$
\begin{aligned}
&T_{1}(n)=T_{1}(n-1)+c n=T_{1}(n-2)+c(n-1)+c n=\ldots\\
&\begin{array}{l}
=T_{1}(3)+c(4+\ldots+n-1+n)=1+(n-3) \cdot \frac{4+n}{2}= \\
=c \cdot \frac{n^{2}+n}{2}-3 c+1=\theta\left(n^{2}\right)
\end{array}
\end{aligned}
$$
Ответ: $\theta\left(n^{2}\right)$

{\bf 2)} Доказать, что  для рекуренты $T_2(n) = T_2(n-1)+4T_2(n-3)$ (при $n > 3$) справедлива оценка $\log T_2(n) = \Theta(n)$ (в асимптотической оценке основание логарифма неважно, поэтому без ограничения общности возьмём его равным 2).\\
Доказазываем по индукции по $ n $:\\
База индукции: $ n = 1,2,3 $ - верно.
Предположение индукции: пусть для $ \forall k \leq n $ - утверждение верно. Докажем для $ k = n + 1 $\\
По предположению индукции имеем соотножения (должны быть разные константы, но от этого суть далнейших выкладок не меняется):\\
$$
c_{2} \cdot n \leq \log T_{2}(k) \leq c_{1} \cdot n
$$
$$
c_{2} \cdot n \leq \log T_{2}(k - 2) \leq c_{1} \cdot n
$$
Получаем оценку на шаг индукции:
$$
T_{2}(k+1)=T_{2}(k)+4 T_{2}(k-2) \leq 2^{c_1k} +4 \cdot 2^{c_{1}(k-2)} \leq 2^{c_{1} k}+\frac{4}{2^{c_1}} \cdot 2^{c_{1} k} \leq 4 \cdot 2^{c_{1} k} \leq 4 \cdot 2^{c_{1}(k+1)}
$$
Если возьмём логарифм от полученного неравенства, получим нужное выражение: $ \log T_2(k+1) \leq C \cdot (k+1) $. Оценка сниза получается также, следовательно, по индукции мы доказали, что $\log T_2(n) = \Theta(n)$.

\subsection*{Задача 5}
Заметим, что значение величины $ m \cdot u + n \cdot v = inv = 2ab$, так как:\\
\begin{itemize}
\item если $ m \geq n $, то значение этой величины: $ (m-n) u+n(v+u)=m u-n u+n v+n u= mu + nv = inv $\\
\item иначе: $ m(u+v)+(n-m) v= mu + nv = inv $\\
\end{itemize}
Получается для любой ветки в цикле наша величина остаётся инвариантной и первоначально равняется $2ab$. После выполнения цикла в одной из переменной $ m $ или $n$ будет лежать значение $ NOD(a,b) $, а вдругой 0. Также известно соотношение: $NOD(a,b) \cdot NOK(a,b) = ab$, следовательно, после выполнения работы в переменной $z$ будет лежать $2NOK(a,b)$.

\end{document} % конец документа